\documentclass[blue,iPad,cn]{elegantnote}

\title{\bfseries 中华乐教导引笔记}

\author{\href{www.liyunzhe.cn}{\itshape 李蕴哲} \\
		Wuhan University}	%\thanks{Elegant\LaTeX{} 其他模板下载地址:\href{https://ddswhu.me/resource/}{https://ddswhu.me/resource/}} }
\date{\today}

\usepackage{listings} 
\lstset{language=[LaTeX]{TeX},basicstyle=\footnotesize\ttfamily}


\begin{document}
{\color{ecolor}{\maketitle}}
% logo
%\centerline{\includegraphics[width=0.25\textwidth]{logo.png}}

\section{引言}

我们每个人是否是充实而快乐?

我们的出路和福祉何在?

山川异域,风月同天。而我们为什么只有“武汉加油”。

\subsection{日本送中国物资寄语}

\textbf{山川异域,风月同天。}这句话具有佛教色彩,但是也同样有东方美学,东方式的情感,体现了东方的诗意思维。\textbf{岂曰无衣,与子同裳},来自诗经。这曾经是中国人很熟悉的表达方法,但是近几十年在国人的表达中渐渐疏离了。表达可以有多种方式,每一种表达都有使用的地方。但是这种古典失落的现象要重视。能雅能俗,能输能白。

\subsection{出路和福祉}

工具理性,科技思维是一种很好的思维。但是人的思维是要理性和感性并重的。

人的思维四个维度:

\begin{itemize}
	\item 智商。现在人基本不缺,甚至会自以为是。
	\item 情商。现在很多人已经在关注情商了,情商的提高做到三个方面:
	\begin{itemize}
		\item 努力突破自我中心,不要过于自我为是
		\item 多做换位思考
		\item 追求悉心体贴
	\end{itemize}
	\item 灵韵。当我们对美和艺术钟爱、在行的时候,灵韵就到了一个较好的状态。
	\item 情怀。东方文化首重情怀,情怀无价。
\end{itemize}

\subsection{真的快乐吗?}

中国文化倡导众美合一。纷吾既有此内美兮。但是也要发现自己的缺点。

重视自己内心的灵性,个体是谦和的,是卑微的,但是自己在气象上可以做到卓尔不群。要更加具有文化人的自觉,规避现代人的傲慢。实现宜古宜今的趣味与风雅。

\subsection{传统文化与文化传统}

文化传统就是一种文化的精气神。传统文化要有扬弃的态度。中华文化传统有雅俗的方面。文化传统的价值怎么估量都不过分,需要一个族群有价值的自觉,要敬畏文化传统。中华的文化传统在近一百多年有不同的际遇,这是一个沉重的话题。

\textbf{党和国家:不忘本来,吸收外来,面向未来。}如何践行,需要我们的考量。文化传统涉及到文化立场。每个人都有文化立场。文化立场,文化格调,文化风骨,是每个文化人都要思考的话题。

很多学者有过思考:

\begin{itemize}
	\item 饶宗颐:万古不磨意,中流自在心。
	\item 叶嘉莹:书生报国成何计,难忘诗骚李杜魂。
	\item 黄会林:大道本无我,青春长与君。
\end{itemize}

通过对文化个案的悉心领会去感受中华文化的本体。散为群星,聚为满月。中华文化就是月映万川,是具体性和独特性的统一。

\subsection{诗乐源流}

在中华文化的观念中,诗乐同源,古乐璀璨。古乐是上古时期的文化样态,常用来祭祀。诗经可以入乐歌唱,如今只有文本,甚至有些只有笙诗。只有到古代的情景下才能将内涵发扬出来。

《关雎》经学家的解读为赞颂后妃之德。在古乐场景下解读,不仅仅是表达对女子的爱慕。《关雎》其实是闹洞房,众人齐唱的歌。既有爱的热烈情思的流露,又有一种揶揄的色彩。

《蒹葭》、《野有蔓草》\footnote{体现了遇到什么美好就是什么的美好祝愿}都是先民的时代,传达的一种美好的祝愿。诗经的篇章中体现出的生命情调,灵感悟性,精神的丰腴。要融通传统与现代,贯通古今融汇有价值的美好的元素。王冲说过“善才有深浅,无有古今”。“文有真伪,武有古心”,要领会古今的文化。中华文化一通百通,是一种辩证统一。

到了汉代,有所分化,但是还有很多融合。诗乐交融是中华文化的一个很重要的色彩。词的本质也是音乐文学。晚唐词都是月前花下,入乐歌唱的。研读文本的时候,要注重文艺语境和文艺特征。

\subsection{中国文化中诗的内涵}

诗者,寺也。诗是把人内心的情愫以语言的形式展现出来。诗,源于情感的悦动。叶嘉莹认为,诗是兴发感动的载体。

诗乃天地之心。每个人喜欢的诗是不一样的。除了很纯美的诗,还有很朴素的诗。这都无妨诗的样貌。因为诗是天地之心,天地是多元的。

不同样态的中国诗:

\begin{itemize}
	\item 中国古代的抒情诗词,比如李商隐的作品《春雨》:	“怅卧新春白袷衣,白门寥落意多违。红楼隔雨相望冷,珠箔飘灯独自归。远路应悲春晼晚,残霄犹得梦依稀。玉珰缄札何由达,万里云罗一雁飞。”文饰之悲,沧海月明珠有泪。李商隐的诗,体现了抒情诗的极致。李清照的词,体现了贵族女性的才华,思维的灵妙。《一剪梅》婉约令词的巅峰状态。中国女性的柔善,两方着笔,“此情无计可消除,才下眉头,却上心头”。王菲演绎成了流行歌。纳兰容若又翻出新意。
	\item 很沉重的诗,比如李绅的《悯农》。杜甫的《又呈吴郎》:“堂前扑枣任西邻,无食无儿一妇人。不为困穷宁有此?只缘恐惧转须亲。即防远客虽多事,便插疏篱却甚真。已诉征求贫到骨,正思戎马泪盈巾。”杜甫的柔善,深沉的人文情怀,能忍常人之不能忍,更加的体恤众生\footnote{李白,不能忍常人之能忍,因此更加天马行空}。清人卢德水说:“杜诗温柔敦厚,其慈祥恺悌之衷,往往溢于言表。如此章,极煦育邻妇,又出脱邻妇;欲开导吴郎,又回护吴郎。八句中,百种千层,莫非仁音,所谓仁义之人其音蔼如也。”
\end{itemize}

数点梅花天地心。意象不再是自然之物了,具有更独特的传情达意的功能。

\begin{example}
	红豆意象。盛唐第一诗人王维的《相思》,红豆便成为了相思子,既浓且艳的美好内涵,再加上悲戚的意味。个体的情感依托更多的社会环境,因此后来更多的表现家国情怀。作者之意未必然,读者之意何必不然。中国文人没有不写红豆的。红楼梦里面宝玉的《红豆曲》。
\end{example}

\begin{example}
	梅花意象。梅花塑造了中国人的文化内涵,是中国的国花之一。孙中山先生说:“唯有梅花真国魂”。陆游的《卜算子咏梅》体现了中国文人的风骨,“零落成泥碾作尘,只有香如故”。对梅花的精神的赞誉。王冕的《墨梅诗》:“不要人夸好颜色,只留清气满乾坤。”高崎:“雪满山中高士卧,月明林下美人来。”“疏影横斜水清浅,暗香浮动月黄昏。”“十年不到香雪海,梅花亦我我亦梅。”
\end{example}

中国人人与自然交融,中华文化的华彩极为充浓。

“东风不来,三月的柳絮不飞。”柳絮有丰富的生命意义,又有漂泊、悲惨融入其中。中国人对诗的理解,呈现的诗意的思维和表达的精妙。

中国诗乐文化的璀璨成果:

\begin{itemize}
	\item 诗词。国诗的概念。
	\item 昆曲。
	\item 古琴。唯一可以称琴道的乐器。琴乃文器之心。
	\item 红楼梦。
\end{itemize}

\subsection{诗化的感性能力和理性能力}

雅人高致。古人强调,修得一颗明心。苏雪林,“人雅不关居有著,鸟鸣如唤克鹈鹕。”深情妙赏之人,往往有些许自恋,往往异常可爱。由于现代的逻辑思维,我们要规避逻辑的束缚,避免欲念的迷狂。中华文化的辩证,通达与和谐。

古人有一种更诗意的感觉。现在,雨水刚过,惊蛰将来,春气发扬。梅花已谢杏花新。古人认为这个是二十四方花性\footnote{每五天有一种新的花开放}的节点。

中国的诗乐塑造了柔善的人格样态,崇刚恋阴,婉约风致。

\begin{example}
	沈祖棻的代表作1932年的《浣溪沙》,张充和1943年的作品《临江仙》,体现了婉约风致。
\end{example}

期待这个课程能成为一段隽永的记忆!

\end{document}
